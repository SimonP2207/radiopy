\documentclass[10pt,a4paper]{article}
\usepackage[utf8]{inputenc}
\usepackage{amsmath}
\usepackage{amsfonts}
\usepackage{amssymb}
\usepackage{graphicx}
\usepackage{lmodern}
\usepackage[left=2cm,right=2cm,top=2cm,bottom=2cm]{geometry}
\renewcommand{\familydefault}{\sfdefault}

\author{Simon Purser}
\title{\texttt{radiopy}: Handling astronomical radio observations with python}
\begin{document}
\maketitle
\section{Introduction}
\label{sec:introduction}

\subsection{Purpose}
\label{sec:purpose}
When we conduct radio astronomical observations of the sky, we usually focus on targets of interest and observe them over specific frequency ranges. These observations yield a large amount of important data in their analysis, for which the \texttt{radiopy} package was initially developed to handle the various analysed properties (as well as their errors).

\subsection{Dependencies}
\label{sec:dependencies}
\texttt{radiopy} requires the following versions/libraries/modules:
\begin{itemize}
	\item Python 3 (tested with version 3.6.7)
	\item numpy (tested with version 1.15.1)
	\item scipy (tested with version 1.1.0)
	\item matplotlib (tested with version 2.2.3)
	\item uncertainties (tested with version 3.0.2)
\end{itemize}

\section{Classes}
\label{sec:classes}

\subsection{The \texttt{Flux} class}
\label{sec:classes;flux}

\subsection{The \texttt{Dflux} class}

\subsection{The \texttt{Size} class}
\label{sec:classes;size}

\subsection{The \texttt{Coordinate} class}
\label{sec:classes;coordinate}

\section{Methods}
\label{sec:methods}

\section{Future Features}
\label{sec:ff}

\subsection{Features in development}
\label{sec:ff;beingdeveloped}

\subsection{Features to be developed}

\label{sec:ff;tobedeveloped}
\end{document}